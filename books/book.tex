% !TEX program = xelatex

\documentclass[12pt]{book}

\usepackage{hyperref}

\usepackage{amsmath,amsfonts,amssymb}
\usepackage{amsthm}
\usepackage{bidipoem}
\usepackage{longtable}

\usepackage{pgf,tikz,pgfplots}
\pgfplotsset{compat=1.15}
\usepackage{mathrsfs}
\usetikzlibrary{arrows}
%\pagestyle{empty}


\usepackage{xepersian}
\settextfont{Yas}
\setdigitfont{Yas}


\defpersianfont\nast{IranNastaliq}
\defpersianfont\sols{XB Sols}
\defpersianfont\yas{Yas}

\title{متن آزمایشی}
\author{محمد رضیئی فیجانی}

\newtheorem{thm}{قضیه}[section]
\newtheorem{Def}{تعریف}[section]
\newtheorem{test}{تمرین}[chapter]
\newcommand{\dd}{\, \mathbf{d} }


\begin{document}
\maketitle
\tableofcontents
\chapter{مقدمه}\label{chpt1}

چشمگیرترین قسمت بالایی آرامگاه، پیکر تراشیده داریوش بزرگ است که ۲.۷۰ متر بلندی دارد و در سمت چپ مجلس، به‌حالت نیم‌رخ و روبه سمت راست، بر سکویی سه‌پلّه‌ای ایستاده‌است. وی تاجی کنگره‌دار بر سر نهاده و ردایی بلند و گشادآستین و پُرچین و شکن به‌تن کرده‌است. کفشش ساده و نیم‌تنه‌اش را کمربندی استوار نموده، ریشش بلند و حلقه‌حلقه‌است و مویش را پیچ‌درپیچ در پشت سر انبوه کرده و آراسته‌است، حلقه‌ای از گوش آویخته و دستبندهایی نقش دار بر مچ دارد. وی با دست چپ یک کمان را گرفته‌است و دست راستش را به حالت نیایش، رو به جلو آورده و با کف و انگشتان به‌ سوی آتشدان و نماد فروهر دراز کرده‌است.

\[
\int f(x) \dd{x}
\]
\section{آزمایش}

توضیحات بیشتر در فصل 
\ref{chpt2}
موجود می باشد.
\begin{thm}
در قضیه فیثاغورث آمده است که :
\end{thm}
\begin{proof}
	با توجه به برهان خلف اثبات شد.
\end{proof}
\begin{Def}
	میبینم که ... 
\end{Def}

\[
	\lim_{x \to \infty} \frac{\cos 2x + \tan 3x}{\ln x + \zeta}
\]

\[
\bar{x} = \frac{x_1 + \dots+x_n}{n}	
\]

\[
L_{i}(x)=\prod_{j=0 \atop j \neq i}^{n} \frac{x-x_{j}}{x_{i}-x_{j}}
\]

\begin{test}
	با توجه به فلان چیز حل کنید.
\end{test}
\begin{proof}[حل]
	به یک طریق مثلا حل شد.
\end{proof}
\chapter{مراجع}\label{chpt2}

چشمگیرترین قسمت بالایی آرامگاه، پیکر تراشیده داریوش بزرگ است که ۲.۷۰ متر بلندی دارد و در سمت چپ مجلس، به‌حالت نیم‌رخ و روبه سمت راست، بر سکویی سه‌پلّه‌ای ایستاده‌است. وی تاجی کنگره‌دار بر سر نهاده و ردایی بلند و گشادآستین و پُرچین و شکن به‌تن کرده‌است. کفشش ساده و نیم‌تنه‌اش را کمربندی استوار نموده، ریشش بلند و حلقه‌حلقه‌است و مویش را پیچ‌درپیچ در پشت سر انبوه کرده و آراسته‌است، حلقه‌ای از گوش آویخته و دستبندهایی نقش دار بر مچ دارد. وی با دست چپ یک کمان را گرفته‌است و دست راستش را به حالت نیایش، رو به جلو آورده و با کف و انگشتان به‌ سوی آتشدان و نماد فروهر دراز کرده‌است.

\renewcommand\poemcolsepskip{1.3cm}
\begin{traditionalpoem}

اگر عالم همه پرخار باشد &&
دل عاشق همه گلزار باشد
\\
وگر بي كار گردد چرخ گردون &&
جهان عاشقان بر كار باشد
\\
همه غمگين شوند و جان عاشق &&
لطيف و خرم و عيار باشد
\\
به عاشق ده تو هر جا شمع مرده ست &&
كه او را صد هزار انوار باشد
\\
وگر تنهاست عاشق نيست تنها &&
كه با معشوق پنهان يار باشد
\\
شراب عاشقان از سينه جوشد &&
حريف عشق در اسرار باشد
\\
به صد وعده نباشد عشق خرسند &&
كه مكر دلبران بسيار باشد
\\
وگر بيمار بيني عاشقي را &&
نه شاهد بر سر بيمار باشد
\\
سوار عشق شو وز ره مينديش &&
كه اسب عشق بس رهوار باشد
\\
به يك حمله ترا منزل رساند &&
اگرچه راه ناهموار باشد
\\
علف خواري نداند جان عاشق &&
كه جان عاشقان خمار باشد
\\
ز شمس الدين تبريزي بيابي &&
دلي كو مست و بس هشيار باشد


\end{traditionalpoem}

\begin{equation}
	\iint \ldots \int_{\mathbf{D}} f\left(x_{1}, x_{2}, \ldots, x_{n}\right) \mathbf{d} x_{1} \mathbf{d} x_{2} \ldots \mathbf{d} x_{n}
\end{equation}

\begin{equation}
	T=\left[a_{1}, b_{1}\right) \times\left[a_{2}, b_{2}\right) \times \cdots \times\left[a_{n}, b_{n}\right) \subseteq \mathbb {R}^{n}
\end{equation}

\begin{equation}
\Gamma_{\epsilon}(x)=\left[1-e^{-2 \pi \epsilon}\right]^{1-x} \prod_{n=0}^{\infty} \frac{1-\exp (-2 \pi \epsilon(n+1))}{1-\exp (-2 \pi \epsilon(x+n))}
\end{equation}

\begin{equation}
\left( \begin{array}{l}{c t^{\prime}} \\ {x^{\prime}} \\ {y^{\prime}} \\ {z^{\prime}}\end{array}\right)=\left( \begin{array}{cccc}{\gamma} & {-\gamma \beta} & {0} & {0} \\ {-\gamma \beta} & {\gamma} & {0} & {0} \\ {0} & {0} & {1} & {0} \\ {0} & {0} & {0} & {1}\end{array}\right) \left( \begin{array}{l}{c t} \\ {x} \\ {y} \\ {z}\end{array}\right)
\end{equation}

{\bf\sols
ثلث


سلام آقا خوبی ؟
}
\sols
\begin{center}
\begin{longtable}{ | c | c | c | }
	\hline 
	\textbf{ردیف} & \textbf{نام}	 &  \textbf{فامیلی}   \\ \hline 
	1 & محمد  & رضیئی \\ \hline
	2 & علی & جان \\ \hline
	3 & محمد  & رضیئی \\ \hline
	4 & علی & جان \\ \hline
	5 & محمد  & رضیئی \\ \hline
	6 & علی & جان \\ \hline
	7 & محمد  & رضیئی \\ \hline
	8 & علی & جان \\ \hline
	9 & محمد  & رضیئی \\ \hline
	10 & علی & جان \\ \hline
	11 & محمد  & رضیئی \\ \hline
	12 & علی & جان \\ \hline
	13 & محمد  & رضیئی \\ \hline
	14 & علی & جان \\ \hline
	15 & محمد  & رضیئی \\ \hline
	16 & علی & جان \\ \hline
	17 & محمد  & رضیئی \\ \hline
	18 & علی & جان \\ \hline
	19 & محمد  & رضیئی \\ \hline
	20 & علی & جان \\ \hline
	21 & محمد  & رضیئی \\ \hline
	22 & علی & جان \\ \hline
	23 & محمد  & رضیئی \\ \hline
	24 & علی & جان \\ \hline
	25 & محمد  & رضیئی \\ \hline
	26 & علی & جان \\ \hline
	27 & محمد  & رضیئی \\ \hline
	28 & علی & جان \\ \hline
	29 & محمد  & رضیئی \\ \hline
	30 & علی & جان \\ \hline
	31 & محمد  & رضیئی \\ \hline
	32 & علی & جان \\ \hline
	33 & محمد  & رضیئی \\ \hline
	34 & علی & جان \\ \hline
	35 & محمد  & رضیئی \\ \hline
	36 & علی & جان \\ \hline
	37 & محمد  & رضیئی \\ \hline
	38 & علی & جان \\ \hline
	39 & محمد  & رضیئی \\ \hline
	40 & علی & جان \\ \hline
	41 & محمد  & رضیئی \\ \hline
	42 & علی & جان \\ \hline
	43 & محمد  & رضیئی \\ \hline
	44 & علی & جان \\ \hline
	45 & محمد  & رضیئی \\ \hline
	46 & علی & جان \\ \hline
	47 & محمد  & رضیئی \\ \hline
	48 & علی & جان \\ \hline
	49 & محمد  & رضیئی \\ \hline
	50 & علی & جان \\ \hline
	51 & محمد  & رضیئی \\ \hline
	52 & علی & جان \\ \hline
	53 & محمد  & رضیئی \\ \hline
	54 & علی & جان \\ \hline
	55 & محمد  & رضیئی \\ \hline
	56 & علی & جان \\ \hline
	57 & محمد  & رضیئی \\ \hline
	58 & علی & جان \\ \hline
	59 & محمد  & رضیئی \\ \hline
	60 & علی & جان \\ \hline
	61 & محمد  & رضیئی \\ \hline
	62 & علی & جان \\ \hline
	63 & محمد  & رضیئی \\ \hline
	64 & علی & جان \\ \hline
	65 & محمد  & رضیئی \\ \hline
	66 & علی & جان \\ \hline
	67 & محمد  & رضیئی \\ \hline
	68 & علی & جان \\ \hline
	69 & محمد  & رضیئی \\ \hline
	70 & علی & جان \\ \hline
\end{longtable}
\end{center}
\chapter{geogebra}
 
 سایت در اینجا
\href{https://www.geogebra.org/download}{geogebra.org}
میباشد.

\definecolor{sexdts}{rgb}{0.1803921568627451,0.49019607843137253,0.19607843137254902}
\begin{tikzpicture}[line cap=round,line join=round,>=triangle 45,x=1cm,y=1cm]
\begin{axis}[
x=1cm,y=1cm,
axis lines=middle,
ymajorgrids=true,
xmajorgrids=true,
xmin=-5,
xmax=5,
ymin=-3.0,
ymax=3.0,
xtick={-8,-7,...,7},
ytick={-3,-2,...,3},]
\clip(-5,-3.0) rectangle (5,3.0);
\draw[line width=2pt,color=sexdts,smooth,samples=100,domain=-5:5] plot(\x,{cos(((\x))*180/pi)});
\end{axis}
\end{tikzpicture}



\end{document}